\section{Conclusion}
\label{sec:conclude}

New NVMM technologies are emerging and they will appear on the processor
memory bus with low-latency, high-bandwidth and byte-addressability. These
properties provide both opportunities and challenges to system sofware
programmers.

In this paper we introduce promising non-volatile memory technologies such as
PCM, STT-RAM and memristor, and show
the NVMM software researches on four categories: file system, programming model,
DRAM extension and caching and acceleration. With NVMM sitting on the
processor memory bus, it does not only allow applications to access directly
to reduce overhead, but also brings design issues such as CPU caching impact
and write reordering. How to conquer these problems without hurting the 
performance NVMM provides is the main task for NVMM software designers. The
paper also introduces our own work to accelerate storage access with NVMM.

Lastly, we discuss miscellaneous NVMM system designs, and try
to explore the future usage of NVMM in computer systems.

\ignore{
NVMMs will require significant changes to systems architectures if applications
are to exploit the performance they can provide.  We have described and
evaluated \Chell{}, a system that uses NVMM to accelerate file access.
\DAChell{} provides direct access to file data for file systems that reside
solely in NVMM, while \CChell{} uses NVMM a persistent, write-back cache of a
conventional file system.  In both cases, \Chell{} allows the application to
access file data, in most cases, without interacting with the filesystem.

We found that \DAChell{} improves 4~kB access latency by up to 19\% relative to
file access through a file system built for NVMM and improves Berkeley-DB
performance by 6\% over PMFS.  \CChell{} can
reduce 4~kB access latency by up to 50\% and
improve Berkeley-DB performance by up to 4.8\x{} relative to file access using the
system's file buffer.

Finally, we demonstrated a novel file IO interface that eliminates the need to
copy data to and from userspace memory.  Since the advent of NVMM will make such copies the dominant cost in accessing file data, eliminating them is necessary if applications are going to fully levelage NVMMs.
}

\ignore{
In this paper we describe two systems use emerging non-volatile maing memory
to accelarate storage accesses.
\Muse{} allows programs to access files stored in fast, non-volatile memories
without going through the operating and file system while still preserving the
protections they provide.  We have shown that \Muse{} can reduce latency for
small accesses by up to 2\x{} and improve bandwidth by up to 4\x{}.  \Muse{}
demonstrates that it is possible to expose most of the performance of these
memories for file access without significant changes to the operating or file
system.

\Chell{} is a caching system using NVMMs to accelerate file access on slower
storage devices.
\Chell{} reduces the operating system and file system overhead
by mapping cache pages to user space and bypassing OS when service cache hits, 
which significantly reduce the cache hit latency. Evaluation shows that \Chell{}
can improve performance over page cache by up to 10\x{} for writes and
1.7\x{} for reads.
}
