\section{File system} 
\label{sec:fs}

To efficiently make use of the NVMM and expose a user-friendly
interface to applications, system software developers need to consider
how to manage NVMM and work with existing space applications smoothly.
As NVMM can be treated as storage device, file system is a straight forward
way to manage the NVMM.

However, although NVMM is non-volatile, its difference from slow storage
devices makes the NVMM file system design needs to consider several special
characteristics:

\ignore{
\begin{itemize}
\item Performance. 
\item Persistency.
\end{itemize}
}

\textbf{Performance} Traditional storage devices are sitting on I/O bus
(like SATA and USB) or high performance system bus such as PCIe. To access
them, applications need to pay the software overhead of mode switch, VFS and
block layer. For slow storage devices the software overhead is negligible,
but with NVMM the story is different. As NVMM has access latency close to
DRAM, the processors should access them directly via load and store
operations to fully exploit the NVMM performance.



\textbf{Consistency} Modern processors may reorder write operations to improve
performance. This is not an issue for DRAM because the cache consistency
protocols provide all the processors with a consistent view of the memory.
However, out-of-order writes to NVMM may leave the software (for example,
file system) in an inconsistent state if a power failure occurs. Software
for NVMM must take caution of the consistency issue. 

To overcome the write reordering issue, several solutions are proposed:

\begin{itemize}
\item Use write-through caching to make sure for each operation
data is flushed to the NVMM. This is a slow approach. 
\item Flush the entire cache when encounter a memory barrier. This solution
adds significant overhead to the performance.
\item Track the dirty cache lines and flush them accordingly. This is much
more efficient than flushing the entire cache, but may need software bookkeeping.
\item Use hardware primitives. This is the most efficient approach, but requires
hardware modifications.
\end{itemize}

Another inconsistency issue comes from the power failure during a single
write operation. If the write operation is not atomic, the data is left
in an inconsistent state. The common approaches to overcome the issue includes:

\begin{itemize}
\item Hardware atomic operations. This solution is simple but may require
hardware modifications.
\item Logging. Logging is a widely-used mechanism in file system and databases
to ensure write atomicity. There are two kinds of logging: Undo logging, which
saves the original value to the log and then update in place; Redo logging,
which write the new data to the log and then copy to the target place.
\item Copy-on-write operation. This is used in tree structure updates. Whenever
update the node, a new node is created and the pointer in the parent node is
updated atomically. This may result in cascade update up to the root.
\end{itemize}
%\textbf{Protection} 


BPFS~\cite{BPFS} is a good example to show how to design a NVMM file system,
and demonstrates the challenges in the implementation. BPFS guarantees that
each file system operation is performed atomically and in order.

BPFS provides these guarantees with software implementation and hardware
primitives. In software part, it uses \emph{short-circuit shadow paging}
to provide fine-grained copy-on-write implementations. It also proposes
two hardware primitives, \emph{atomic 8-byte writes} and \emph{epoch barriers}
to support the short-circuit shadow paging and improve the file system
performance.

Shadow paging is a common file system design to keep file system in consistent
state during system failure. To update data with shadow paging, file system
does not write new data in-place; instead, it makes a copy of the old data,
update the copy with new data, and then update the pointer in the upper layer
atomically to point to the new data. Shadow paging guarantees the update is done
atomically and file system is always in a consstent state.
 ZFS~\cite{zfs} and WAFL~\cite{wafl} are two file
systems that use shadow paging. However, traditional shadow paging cascades
the copy-on-write operation from the modification place up to the root of 
the file system tree, and hurts performance. With short-circuit shadow paging,
BPFS does not always need to propagate the change up to the root, and performs
shadow paging at fine granularity to improve the performance.

To short-circuit the shadow paging, BPFS uses the atomic 8-bytes write primitive
to update the pointer, so that the pointer update in the shadow paging is done
atomically, eliminates the need to cascade the copy-on-write operations.
8-byte is the size of pointer in 64bit systems, so if file system only needs to
update a pointer in a data block, then the update is atomic. To support
atomic 8-byte writes, BPFS adds a capacitor to each DIMM, so that it has enough
power to finish all the outstanding writes in the row buffers when the system
power fails.

To provide safety and consistency, file system also needs to ensure that write
is committed to persistent storage in program order. However, existing cache
architecture and memory controller may reorder the writes to DRAM to improve
performance. Operating systems provide \texttt{mfence} primitive to guarantee
that all CPUs have a consistent view of the memory, but it does not impose
any constraints on the write order to main memory. A NVMM file system must
provide write ordering to file system updates.

A simple way to provide write ordering is to flush the CPU cache line after
write operation, or uses write-through to bypass CPU cache directly. However,
these operations are expensive and undermines performance. Instead, BPFS uses
epoch barrier to resolve the issue. An epoch is a batch of write operations,
and hardware guarantees that epochs are committed to NVMM in program order.
With the epoch barrier all the current writes cannot proceed until all the
writes in the previous epochs are committed to NVMM and file system consistency
is maintained. BPFS implements the epoch barrier primitive by adding counters,
bits and capacitors to CPUs.

With hardware support, BPFS implements a transactional file system for NVMM
and gains better performance than traditional file systems.

SCMFS~\cite{scmfs} is a file system designed for NVMM that reuses the memory
management module in the operating system to reduce the implementation
complexity. SCMFS aims to minimize the CPU overhead of the file system
operations, and also assign each file to a contiguous virtual address space
to simplify the read/write file operations.

SCMFS adds a new address range to the operating system for the files resided
on the NVMM, and 
adds a new memory zone \emph{ZONE\_STORAGE} for NVMM space allocation. The
NVMM allocation function \texttt{nvmalloc()} comes from \texttt{vmalloc()}
so that the allocated pages have contiguous virtual address but not necessary
to be contiguous in physical address.

The author claims that running in virtual address helps to make the SCMFS
implementation simple and improve the performance, but it also increases
TLB misses. To address the issue of write reordering, SCMFS does not use the
hardware primitives proposed by BPFS, instead it combines the Linux kernel
\texttt{mfence} and \texttt{clflush} operations to flush the CPU cache in
range and sequence the write operations. It may not be as fast as hardware
primitives, but the performance is acceptable for common workloads.

SCMFS stores all the virtual-physical page mapping information in the NVMM
to restore the mappings when system fails and reboots. However this is not
a scalable solution: the mapping table size increases with the file system
size, and the time to mount the entire file system also increases. SCMFS
resolves the issue with a lazy mapping, which maps only the metadata and inode
table during the mount, and file data is mapped in background.
Another scalability issue of SCMFS is that it has a fixed space for inode table,
which means the number of files it support is fixed, and it does not scale
well with NVMM size.

An evolution of SCMFS~\cite{superpage} further improves SCMFS performance
by allocating larger pages and reducing TLB misses. Superpage~\cite{superpage}
divides NVMM into two regions, one of normal page allocation (4KB) and the other
for superpages (2MB). By allocating normal pages for small files and file 
metadata while allocating superpages for larger files, file system performance
improves with less TLB misses.



PMFS~\cite{PMFS} is another NVMM file system. It is aiming to provide a
light-weight POSIX file system that exploits NVMM's byte-addressability to
avoid overheads of block layer and enable direct NVMM access by application.
PMFS only requires a simple hardware primitive to provide software enforceable
guarantees of durability and ordering of stores to NVMM and exposes the 
performance potential of NVMM to user applications.

The author analyzes several approaches to enforce consistency, and finds out
using NVMM as a write-back cacheable memory and explicitly flushing modified
data from CPU cache (\emph{clflush} works well. However, current \emph{clflush}
implementation is strongly ordered and incurs performance penalty. Instead,
PMFS proposes a optimized \emph{clflush} implementation which weaker ordering
and hence improve performance. To provide the write ordering PMFS proposes
a hardware primitive, \emph{pm\_wbarrier} to guarantee durability of NVMM stores
already flushed from CPU caches.

As a file system, PMFS also provides consistency. BPFS uses a file-grained
shadow paging mechanism, while PMFS uses logging. There are two kinds of logging
in file system implementation, redo and undo logging. Undo logging in PMFS
requires a \emph{pm\_wbarrier} for each log entry while redo logging requires
only two \emph{pm\_wbarrier} in a single transaction. However, undo logging
is easy to implement and PMFS uses undo logging. Whenever there is an update
to the PMFS metadata, PMFS copies the original value to the log entry and
make it persistent before write the new value in-place. PMFS uses the
8-byte and 16-byte atomic updates of CPU to perform atomic in-place updates.
For file data updates, PMFS uses copy-on-write mechanism.

PMFS also supports write protection. While PMBD uses private mapping to
protect the NVMM from stray writes, it adds complexity to the implementation
for not able to provide a fix virtual address mapping. Instead, PMFS leverages
the CPU's write protect control mechanism (CR0.WP) to open small short write
windows without changing the page tables and hence avoid the expensive
TLB flushes.

To expose the high bandwidth and low latency of NVMM to applications,
PMFS uses the eXecute-In-Place (XIP) interface to support memory-mapped
IO operations. With XIP support the NVMM pages are directly mapped into
application's address space and bypass the page cache and block layer.
In contrast, BPFS does not support \texttt{mmap}, and PMBD transfers
data via page cache, significantly undermines the NVMM performance.
PMFS also supports large page size to decrease the page table size and
TLB entries. ~\cite{mmfs} is another implementation of NVMM file system
whichi supports XIP interface. It integrates memory management with file system
to manage NVMM, allocate NVMM in pages and directly map to application
address space to gain better performance.
