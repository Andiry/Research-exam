\section{Introduction: Quill}
\label{sec:introduction-quill}

\begin{comment}
Emerging non-volatile memories such as spin-torque transfer, phase change, and
memristor-based memories promise to revolutionize IO performance and how
systems manage and provide access to persistent state.  The most aggressive
proposals for integrating these technologies place them on the processor's
memory bus alongside or replacing conventional DRAM.

Placing fast, non-volatile memories on the memory bus raises a host of design
issues, and several proposals examine extensive changes to the file
system~\cite{BPFS} and the basic abstractions that applications use to
manipulate non-volatile data~\cite{mnemosyne}.  These proposals provide
useful features (including transactional semantics), but they require
significant changes to hardware, the file system, and/or the applications to
leverage them.
\end{comment}

\emph{To be combined with introduction to Chell.}

We propose a simpler approach that will expose most of the
performance that these technologies offer while requiring only minor changes to
the file and operating systems and no changes to the application.

Our system, called \emph{\Muse{}}, takes advantage of the fact that these fast
non-volatile memories will appear as pages in the processor's physical address
space.  \Muse{} relies on a conventional file system (in our case ext2) to
manage file system metadata, allocate and release file storage, and set
protection policy.  \Muse{} provides an alternative to the normal system
call-based interface by interposing on file access system calls.  To access a
file, \Muse{} maps the physical pages of non-volatile memory that represent the
file directly into the application's virtual address space.  As a result, no
paging is necessary.  Consequently, for most accesses, applications see both
the latency and bandwidth of DRAM while using a normal file-based interface.

\Muse{} comprises two components.  The first is a generic file IO interposition
layer called \emph{\Switch{}} that allows a library to take over file IO
operations for specific file descriptors and forward requests to a \backend{}
that implements file access functions.  The second component is a \Switch{}
\backend{} that implements the \Muse{} functionality.

This paper describes \Muse{} and \Switch{} in detail and evaluates the impact
of using memory-mapped access to expose the performance that fast non-volatile
memories can provide.  We find that \Muse{} can improve performance by up to
7\x{} compared to accesses through the file system to the same fast
non-volatile memory.

We also describe the limitations of \Muse{} and the approach it takes.  There
are several potential incompatibilities between \Muse{} and standard POSIX file
semantics, and there several ``features'' of POSIX that introduce significant
overheads in \Muse{}.  Our work suggests that some small changes (or
extensions) to the standard interface would make them a better fit for emerging
fast non-volatile memories.

The remainder of this paper is organized as follows.  Section~\ref{sec:system}
describes \Muse{}, \Switch{} and the changes required in the kernel to support
\Muse{}.  Section~\ref{sec:results} evaluates \Muse{}, and
Section~\ref{sec:related} places \Muse{} in the context of other projects.
Section~\ref{sec:conclude} summarizes our conclusions.
