\section{System design} 
\label{sec:systemdesign}

~\cite{systemimplications} shows the implications for a NVMM-only
system. It addresses several changes and challenges to current system design,
such as paging and application running model. A NVMM-only system may benefit
from the persistency of NVMM for fast recovery, but also meets the security
and privacy issue which current system does not have. How to design a
NVMM-only system is still in exploration.

To overcome the CPU cache flush and write re-ordering issues, researchers
have proposed various solution, such as explicit CPU cache flushing, logging
and hardware primitive support. However, these solution are difficult
to implement with legacy software and hardware, and they also add extra
overheads. Some researches try to find a simple and low-overhead solution.

\textbf{WrAP}~\cite{WrAP} proposes a system architecture
 that provides persistence
ordering, persistence atomicity and persistence protection guarantees. 
WrAP consists of three components: a victim persistence cache (VPC); a log
area of NVMM that keeps log of update operations and an asynchronous channel
that used to propagate slow log records to NVMM. WrAP provides a lightweight
firewall prevents arbitrary writes to NVMM and a non-intrusive interface
for interaction between the cache system and NVMM.

Every write to NVMM in WrAP results in two update paths: an atomic update to
the log area in NVMM, and a normal store instruction to the NVMM address. Both
paths are performed simultaneously. VPC hold PCM entries that are evicted
frpm the last-level cache and serves as the final backing store for evicted
variables. VPC does not need to be persistent.
Finally, the update to the home locations will be made
independently by a COPY module that is part of the WrAP controller. 
By propagate the write operation in two paths, WrAP not only makes the 
write atomic, but also avoid the overhead of CPU cache flushing.
Unfortunately WrAP does not provide evaluation of their system.

SoftWrap~\cite{softWrAP} is the software implementation version of WrAP.
Instead of applying a WrAP controller and VPC, SoftWrAP installs a library
to transform applications to use the WrAP APIs to access the NVMM. WrAP
uses VPC to hold PCM entries evicted from CPU cache, while SoftWrAP
uses alias update, routes the cache entries to write to a different location
without affecting the original data. SoftWrAP achieves the same persistent
guarantee as WrAP without hardware modification.

