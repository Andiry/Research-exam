\section{System design} 
\label{sec:systemdesign}

To overcome the CPU cache flush and write re-ordering issues, researchers
have proposed various solution, such as explicit CPU cache flushing, logging
and hardware primitive support. However, these solution are difficult
to implement with legacy software and hardware, and they also add extra
overheads. Some researches try to find a simple and low-overhead solution.

WrAP~\cite{WrAP} proposes a system architecture that provides persistence
ordering, persistence atomicity and persistence protection guarantees. 
WrAP consists of three components: a victim persistence cache (VPC); a log
area of NVMM that keeps log of update operations and an asynchronous channel
that used to propagate slow log records to NVMM. WrAP provides a lightweight
firewall prevents arbitrary writes to NVMM and a non-intrusive interface
for interaction between the cache system and NVMM.

Every write to NVMM in WrAP results in two update paths: an atomic update to
the log area in NVMM, and a normal store instruction to the NVMM address. Both
paths are performed simultaneously. VPC hold PCM entries that are evicted
frpm the last-level cache and serves as the final backing store for evicted
variables. VPC does not need to be persistent.
Finally, the update to the home locations will be made
independently by a COPY module that is part of the WrAP controller. 
By propagate the write operation in two paths, WrAP not only makes the 
write atomic, but also avoid the overhead of CPU cache flushing.
Unfortunately WrAP does not provide evaluation of their system.

WSP~\cite{WSP} stands for whole-system persistence. It aims at systems where
all memory is NVMM, and it transparently recovers an application's entire state,
making a system failure appear as a suspend/resume event. WSP does not
do anything about logging or explicit CPU cache flushing; instead, it does 
\emph{flush-on-fail}, which flushes CPU registers and cache lines to NVMM
on system failure, using a small residual energy window provided by system
power supply. As WSP does not flush during normal operations, it eliminates
the runtime overhead of flushing CPU cache lines on each update entirely, also
it does not require any software modification such as NV-heaps~\cite{nvheaps}
 and Mnemosyne~\cite{mnemosyne}.

WSP consists of NVMM devices, a hardware power monitor, and software suspend/resume routines. When the WSP system fails, the hardware power monitor triggers
an interrupt to the CPU core. All the CPU cores flush their cache lines to the 
NVMM and then halts. On the recovery, resume routine does the inverse to recover
the system to the consistent state. As WSP eliminates the runtime persistency
operations, it shows good performance comparing to NV-heaps.

A remaining issue of WSP is device recovery. As the device drivers are restored
but the devices lost power and need to re-initialize, the state of software
and hardware is not consistent. WSP proposes to clean up device state on
the resume path, or virtualize the devices by using a VM hypervisor.


