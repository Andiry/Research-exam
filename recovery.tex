\section{Recovery} 
\label{sec:recovery}

Several early researches focus on how to recovery the system with NVMMs.
\textbf{RVM}~\cite{RVM} is an efficient,
 portable and easily used implementation of
recoverable virtual memory for Unix enviorments. By eliminating the support
for nesting, distribution and concurrency control, the implementation of 
RVM becomes very simple and easy to use. RVM manages recoverable memory in
segments, each with a file or a disk partition as backing store. Applications
calls the interfaces provided by RVM to begin and commit transactions. RVM
uses a undo/redo logging strategy to make the persistency guarantee. Old values
are copied to undo log, while new values are written to the redo log. Applications call log control commands to force redo log to write into the disks
and become persistent.

\textbf{WSP}~\cite{WSP} stands for whole-system persistence.
 It aims at systems where
all memory is NVMM, and it transparently recovers an application's entire state,
making a system failure appear as a suspend/resume event. WSP does not
do anything about logging or explicit CPU cache flushing; instead, it does 
\emph{flush-on-fail}, which flushes CPU registers and cache lines to NVMM
on system failure, using a small residual energy window provided by system
power supply. As WSP does not flush during normal operations, it eliminates
the runtime overhead of flushing CPU cache lines on each update entirely, also
it does not require any software modification such as NV-heaps~\cite{nvheaps}
 and Mnemosyne~\cite{mnemosyne}.

WSP consists of NVMM devices, a hardware power monitor, and software suspend/resume routines. When the WSP system fails, the hardware power monitor triggers
an interrupt to the CPU core. All the CPU cores flush their cache lines to the 
NVMM and then halts. On the recovery, resume routine does the inverse to recover
the system to the consistent state. As WSP eliminates the runtime persistency
operations, it shows good performance comparing to NV-heaps.

A remaining issue of WSP is device recovery. As the device drivers are restored
but the devices lost power and need to re-initialize, the state of software
and hardware is not consistent. WSP proposes to clean up device state on
the resume path, or virtualize the devices by using a VM hypervisor.


