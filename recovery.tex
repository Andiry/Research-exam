\section{Recovery} 
\label{sec:recovery}

Several early researches focus on how to recovery the system with NVMMs.
RVM~\cite{RVM} is an efficient, portable and easily used implementation of
recoverable virtual memory for Unix enviorments. By eliminating the support
for nesting, distribution and concurrency control, the implementation of 
RVM becomes very simple and easy to use. RVM manages recoverable memory in
segments, each with a file or a disk partition as backing store. Applications
calls the interfaces provided by RVM to begin and commit transactions. RVM
uses a undo/redo logging strategy to make the persistency guarantee. Old values
are copied to undo log, while new values are written to the redo log. Applications call log control commands to force redo log to write into the disks
and become persistent.

Rio~\cite{riofilecache} uses battery-backed memory to build a operating system
with better performance and able to recovery from crash. To prevent the file
cache from corrupt during system fialure, Rio turns off the write-permission
bits in the page table for all file cache pages, only enable it during page
writing. Rio also disables the ability of the processor to bypass the TLB,
enforce the protection for all the pages.

Rio uses the warm reboot mechanism to rebuild the system. When the system 
recovers from failure, it read the file cache extents present in physical memory
and update file system with the data. To figure the file cache contents,
Rio also needs to store the metadata for the file cache. Rio allocates a
separate region of memory called registry to store the metadata information.

Rio improves the system performance by eliminating the sync and fsync system
calls, which are expensive in the original system. However, as buffer cache
is persistent now, the write ordering is essential to make sure the system
works correctly. Rio does not explain how it works on write ordering and CPU
caching impacts.

Another contribution of Rio is that it proposes several fault models of system
failure to memory and tested the system with thousands of reboot. This makes
it more reliable.
 
Rio Vista~\cite{riovista} is a combination of RVM and Rio. It is an evolutionarywork of Rio, and relate closely to RVM design. Like RVm, Rio Vista is a
user-level, recoverable-memory library, and it runs on Rio. Unlike RVM,
which has undo log and redo log, Rio Vista just has one undo log. When
application starts a transcation, Rio Vista copies the origial data to the 
undo log, which is kept in NVMM provided by Rio, and then application can
write new data in-place and discards the undo log by the end of the transaction.

Rio Vista is trying to be simple, handles only the basic transaction features
of atomicity and persistency. Like Rio, it does not address the write ordering
issue and CPU caching issue, which undermines its credability. 

