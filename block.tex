\section{Block device} 
\label{sec:bd}

As NVMM is sitting on the system memory bus and expose the memory range
to the applications, how to protect it from stray writes become a issue.
Also, the system manages the NVMM needs to provide ordered persistence
and be compatible with existing applications using POSIX I/O interfaces.
All these requirements make the NVMM system software complex and difficult
to implement.


PMBD~\cite{PMBD} is an attempt to leverage the existing modules that the 
operating system provides and makes the system light-weight and easy to
implement. Rather than the accessing models and file systems we have seen,
PMBD treats NVMM as a block device sitting on the memory bus. The advantage
is that it is easy to provide protection and order persistence for
a block device than a memory device.

PMBD proposes several solutions to protection and order persistence and
compare the performance. For protection, one solution is to disable page
table entry for write until needed; however, this incurs big penalty on
TLB flush and performance is bad. Buffer the write requests and commit
them as a batch helps to improve the performance for sequential writes,
but not for random writes. The author finds out the best solution is 
\emph{private mapping}: NVMM pages are mapped to virtual space only when
required and unmapped after write finishes. Evaluation shows private
mapping provides 90\% bandwidth of the optimal case. Private mapping
addes overhead for each write request, but reduces page table size and
TLB pollution.

To guarantee write order, PMBD compares two solutions: use a non-temporal
store (\texttt{movntq}) to bypass the CPU cache and \texttt{sfence} to
flush the buffered data, or use \texttt{clflush} to flush CPU cache
and \texttt{mfence} to ensure the data is written back. With micro-benchmark
tests \texttt{movntq/sfence} generally works better than \texttt{clflush/mfence}
solution.

PMBD is targeting at a light-weight system which provides good 
protection of NVMM. Thus, although PMBD provides several optimizations
to improve the performance, the performance is not comparable to tmpfs
and ramfs, which are file system for volatile memories. The fact that
PMBD treats NVMM as a block device undermines the performance of NVMM,
as Linux block layer is designed for slow storage devices and incurs
significant software overhead for fast NVMM. Also, PMBD works with page
cache, which is redundant: as NVMM is sitting on the memory bus, what is
the benefit to copy the data to DRAM first? PMBD does not answer these
questions, but it provides some insights on NVMM protection, which is
not addressed by BPFS.

