\section{Related Work} 
\label{sec:related}

The emergence of fast, non-volatile memories and the prospect of
widely-deployed NVMM has motivated many projects.

BankShot~\cite{BankShot} implements a fast SSD caching system
and allows applications to access the SSD cache directly without going to
kernel, and implements file system permission check in hardware to provide
protection to files. It also supports dirty data tracking and LRU eviction
in hardware.

eNVy~\cite{eNVy} proposes a system that attaches flash to the memory
bus, and using SRAM to achieve high bandwidth and low latency.
WSP~\cite{WSP} utilizes NVMM's persistent property, flushes processor registers
and caches to NVMM on power failures to reduce the system recovery time.
Some vendors have produced non-volatile main memory devices, they are
either small capacity and byte-addressable~\cite{micron-nvdimm},
or large capacity but block-addressable~\cite{smart-system}.
With emerging non-volatile technologies, we can expect
large capacity, byte-addressable NVMMs to appear.

Several systems have exposed non-volatile memories on the processor bus
and provide novel interfaces to the data. Rio Vista~\cite{riovista} and
recoverable virtual memory~\cite{RVM} are among the oldest systems in this
class. Rio Vista exposed raw battery-backed DRAM to the application but made
no provisions for file-based access. RVM provided a simple transactional
interface to raw non-volatile memory.  More recently, several projects~\cite{nvtm,
mnemosyne,hpnvdata} produced frameworks for storing persistent objects
in these memories. Like \Chell{}, these schemes expose persistent memory directly
to the applications, but they also require programmers to rewrite applications
to take advantages of them.

Some researchers integrate NVMM into existing systems with minimal impacts.
PMBD~\cite{PMBD} leverages Linux kernel's block layer and presents NVMM as
a block device to users. It protect the NVMM from stray writes by dynamically
mapping NVMM pages into the kernel space only when needed, and uses non-temporal
store (e.g., \texttt{movntq} and \texttt{clflush} in write-mode to provide write
ordering.  Although PMBD provides protection and ordered persistence,
implementing NVMM as a block device undermines the low-latency and
high-bandwidth properties of NVMM, as the performance is restricted by the
block layer, which is designed for slow storage like hard disks. \DAChell{}
and \CChell{} works as a memory and allow applictions to access NVMM directly
from user space to minimize the software overhead.

A great deal of research focuses on the file system architect for NVMMs.
SCMFS~\cite{scmfs} utilizes the OS VMM module to perform  block
management and layout for files in large contiguous virtual address regions.
BPFS~\cite{BPFS} is a file system designed for NVMM that provides
tranactional guarantees. BPFS uses copy-on-write and eight-byte 
atomic updates to provide metadata and data consistency. However, BPFS does
not support \texttt{mmap()}, and it relies on a hardware approach (epochs)
to support store durability and ordering, which requires hardware
modifications.  PMFS~\cite{PMFS} is a light-weight
file system that exploits persistent memory's byte-addressability, avoiding the
overhead of accessing memory via the block layer and enabling applications to
access persistent memory directly.  It also provides software-enforcebale guarantees of durability
and ordering of writes while protecting persistent memory from stray writes.
Unlike BPFS, PMFS utilizes pm\_wbarrier primitive to flush PM stores to a power
fail-safe destination, and does not require hardware support.

\ignore{However, reads and writes to BPFS and PMFS need to go to kernel, while \Chell{}
eliminates this overhead.}

Both \DAChell{} and \CChell uses \texttt{mmap()} to accelerate NVMM access.
Failure-atomic msync~\cite{atomicmsync} proposes a atomic commit of memory
pages change for \texttt{mmap()ed} files, by flushing memory pages to persistent
store first and then write back to the destination. For \DAChell{} and \CChell{}, as they are running on NVMM, there is no need to write back the data, and
\texttt{msync()} is simely a CPU cache flush.

\begin{comment}
We see two potential future improvements for \Chell{}. First, we hope to improve
it by adding
strong consistency and protection from malicious writes. Second, at present 
\Chell{} cannot avoid
\texttt{memcpy()} entirely, because POSIX APIs provide user buffers to fill with
data. If we modify the POSIX interfaces that return cache pages to applications
directly we can eliminate \texttt{memcpy}.
\end{comment}
