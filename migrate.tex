\section{Page migration} 
\label{sec:migration}

Current non-volatile memory technologies, such as phase change memory
~\cite{PCMHierarchy}, STT-RAM and memristors, all suffer from slower writes and
limited endurance comparing to DRAM.
Thus, how to utilize NVMM's persistency property while reduce the writes to
NVMM to hide the write latency and extend endurance becomes an interesting
topic and several researches contributes to this area.

In the hybrid system installed with both DRAM and NVMM, how to allocate
pages to exploit the performance of these two memories is a question.
\cite{pcmalloc} allocate heap and stack segments with DRAM pages, and allocate
NVMM pages to other segments, because heap and stack segments are updated
frequently, while text segment is read only, and bss/data segments are
updated infrequently. These allocation mechanism allocates write-intensive
pages in DRAM and reduce the writes to NVMM. \cite{pcm3d} combines the DRAM
and NVMM ares in a large flat memory region and migrate write-intensive pages
to DRAM. The migration is performed in operating system and target the
frequently written pages, while leave the read-intensive pages in NVMM.

RaPP~\cite{RaPP} implements the page migration inside the memory controller
to monitor access patterns and migrate pages between NVMM and DRAM. Rapp
proposes a complicate memory controller design with a rank-based page
placement policy. Depending on the access patterns to pages, RaPP create
multiple page queues to store different rank page frames, place
performance-critical pages and frequently written pages in DRAM, place
non-critical pages and rarely written pages in PCM and spread writes
to PCM across many physical frames. By putting the page placement policy
inside memory controller, RaPP achieves better performance than software
solution, but with more sophisticated hardware design.

