\section{Background}
\label{sec:background}

Three emerging non-volatile memory technologies promise to provide fast,
 non-volatile, byte-addressable memories: spin-torque transfer RAM (STT-RAM),
phase change memory (PCM), and memristor-based memories. In the following
sections we briefly introduce each of them.

\textbf{Phase Change Memory} is a promising non-volatile byte-addressable
memory technology. PCM is built with Chalcogenide-based materials, which has
two states, crystalline state and amorphous state. By
injecting current to heat the memory cell to a high temperature for a period
of time, the material crystallize and the resistence reduce. To reset the cell,
apply a large enough current for a short period to melt down the
material and it transition into amorphous state with higher resistence.
 These two states can be read as "1" and "0". As transition into crystalline
state requires longer current injection, PCM has slower writes comparing to
reads.  
As heating causes contraction of the cell, PCM has limited write endurance
up to $10^8$ cycles.
 
\textbf{Spin-Torque Transfer RAM} STT-RAM uses magnetic properties of 
certain materials and electric charges to enable different persistent states 
in memory cells. Each cell of STT-RAM has two ferromagnetic plates, with
one holds a particular polarity permanently, while the other can change the
polarity. If the two plates have the same polarity, the cell stores "1",
otherwise it stores "0". By applying a current STT-RAM can change the polarity
of the plate with soin-transfer torque.
STT-RAM promises potential of unlimited write cycles, and
current prototype using tunneling magnetorresistence showed 32ns for
read latency and 40ns for write latency, with 0.3mA power consumption
per cell~\cite{sttram}. The disadvantage of STT-RAM is shrinking cell size is
extremely difficult.

\textbf{Memristor memory} is a resistive RAM device. Applying electric field
changes the resistence of the memristor device, and different resistence level
can be read as 0 or 1. Recent research on memristor indicates that its
performance could be comparable to DRAM, with write endurance up to $10^{10}$
cycles. Currently memristor is still under development in the laboratory.
 
\begin{table}
	\resizebox{\linewidth}{!}{
	\centering
	\begin{tabular}{|l|c|c|c|c|}
	\hline
%	\multirow{4}{*}{}&PMFS &\DAChell{}\\\hline
%	&\multicolumn{2}{c|}{(Ops per second)}\\\hline
	Item & DRAM & PCM & STT-RAM & Memristor \\\hline
	\hline
	Byte Addressable & Yes & Yes & Yes & Yes \\\hline
	Non-volatile & No & Yes & Yes & Yes \\\hline
	Read Time (ns) & 10 & 10-50 & 10-35 & ? \\\hline
	Write Time (ns) & 10 & 50-500 & 10-90 & ? \\\hline
	Endurance & $10^{15}$ & $10^{9}$ & $10^{15}$ & ? \\\hline
	\end{tabular}}
	\vspace*{1mm}
	\caption{\figtitle{Properties of Non-volatile memory technologies}}
	\label{table:techtrend}
\end{table}

\subsection{Non-volatile memory on memory bus}
\label{sec:background}

When non-volatile memory emerges, there have been researches about how to
integrate it into the system. They can be used to build solid state drives
instead of NAND flash, and NVM Express~\cite{NVMe} targets at accessing
non-volatile memory devices attached through the PCIe bus. However, due to
its byte-addressable and low-latency, high-bandwidth resemble to DRAM,
put them on the processor memory bus become the most reasonable solution
to fully exploit the performance of non-volatile memory.

Although using NVMM as system memory and access them like DRAM is simple,
this approach does not utilie the persistence property of NVMM. Also, the
high write latency and limited endurance comparing to DRAM makes NVMM not
suitable to replace DRAM direcly, and system software is necessary to
manage the NVMM.

Numerous researches propose different system software designs to
manage the NVMM. In this paper we introduce some of the solutions and describe
them in details in the following sections.


\begin{itemize}
\item \textbf{DRAM extension} With the low power comsumption and high density,
NVMM is a good extension to DRAM to increase the system memory size. These
researches focus on how to migrate memory pages between DRAM and NVMM.

\item \textbf{File system} With the persistence property NVMM is suitable for
user data storage, and manage it with a file system to provide applications
fast access is a good way to utilize NVMM. NVMM file system is compatible
with POSIX API, does not require software modification, but need to take care
of the consistency issue.

\item \textbf{Access model} Although NVMM file system is capable of exposing
the NVMM to the applications, the mode switch and VFS layer still adds
overhead. Some papers are investigating fast and safe accessing models of
NVMM, grant applications direct access to the NVMM. These solutions have
good performance but require software modifications.

\item \textbf{Caching and acceleration} With low-latency and high-bandwidth,
NVMM is able to work as a consistent cache for slower storage devices. As
DRAM is largely used as page cache today, using NVMM as cache has additional
benefits that it is persistent and does not to sync to disk as page cache
does.

\end{itemize}  

