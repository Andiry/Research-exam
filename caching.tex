\section{CPU caching} 
\label{sec:caching}

Some researchers~\cite{CPUcaching}
show the impact of CPU caching on the NVMM performance
and programming. Modern CPUs have store buffer to temporarily keep each store
to hide the latency of cache, and WC buffer to batch the writes before
flush to memory. Both store buffer and WC buffer are aiming to hide latency
and improve memory performance, but they also introduce problems on the
persistency of NVMM, and require programming changes on NVMM.

CPU uses write-back mode by default. In this mode stores are kept in cache
before writing to memory. To guarantee stores are committed to NVMM to achieve
persistency, applications must explicitly flush the cache lines for data
updates. This approach is not only error-prone~\cite{singlelock}
but also incurs significant
performance penalty. In contrast, write-through mode removes unnecessary
cache line flushes and reduce the software complexity. \cite{CPUcaching}
compares the two 
modes, and evaluation result shows there is no universally dominant strategy,
but overall write-through mode appears most competitive.
