\section{Caching and acceleration} 
\label{sec:caching}

As NVMM outperform traditional storage devices with low latency and high
bandwidth, build a persistent cache for slower storage devices and
accelerate the file access is a good usage of NVMM. 

\textbf{Rio}~\cite{riofilecache} uses battery-backed memory to build a
operating system
with better performance and able to recovery from crash. To prevent the file
cache from corrupt during system fialure, Rio turns off the write-permission
bits in the page table for all file cache pages, only enable it during page
writing. Rio also disables the ability of the processor to bypass the TLB,
enforce the protection for all the pages.

Rio uses the warm reboot mechanism to rebuild the system. When the system 
recovers from failure, it read the file cache extents present in physical memory
and update file system with the data. To figure the file cache contents,
Rio also needs to store the metadata for the file cache. Rio allocates a
separate region of memory called registry to store the metadata information.

Rio improves the system performance by eliminating the sync and fsync system
calls, which are expensive in the original system. However, as buffer cache
is persistent now, the write ordering is essential to make sure the system
works correctly. Rio does not explain how it works on write ordering and CPU
caching impacts.

Another contribution of Rio is that it proposes several fault models of system
failure to memory and tested the system with thousands of reboot. This makes
it more reliable.
 
\textbf{Rio Vista}~\cite{riovista} is a combination of RVM and Rio.
It is an evolutionarywork of Rio, and relate closely to RVM design.
Like RVM, Rio Vista is a
user-level, recoverable-memory library, and it runs on Rio. Unlike RVM,
which has undo log and redo log, Rio Vista just has one undo log. When
application starts a transcation, Rio Vista copies the origial data to the 
undo log, which is kept in NVMM provided by Rio, and then application can
write new data in-place and discards the undo log by the end of the transaction.

Rio Vista is trying to be simple, handles only the basic transaction features
of atomicity and persistency. Like Rio, it does not address the write ordering
issue and CPU caching issue, which undermines its credability. 

\textbf{Chell}~\cite{Chell} is the system we built that accelerate storage
access with NVMM. Chell can be configured to work in two modes: \DAChell{}
and \CChell{}. \DAChell{} accelerates access to files stored in file systems
that reside solely in NVMM, while \CChell{} uses NVMM as a cache for the
slower storage devices. Below we describe the architecture and implementation
of \DAChell{} and \CChell{}.

\subsection{\DAChell{}}

\cfigure[Figures/ChellD-Stack.pdf,{\figtitle{\DAChell{} architecture}: by combining NVMM with the ability to map files into userspace, \DAChell{} provides applications with a fast, POSIX-compliant interface to data stored in NVMM.},fig:dachell]


\DAChell{} works with a NVMM file system, gives applications direct access to
a file's contents without
requiring any interaction with the operating system in the common case.  As a
result, it can eliminate both the system call overhead required to enter the
kernel and the file system overhead required to locate stored data and perform
permission checks.
Figure~\ref{fig:dachell} shows the high-level architecture of
\DAChell{}. An NVMM-aware file system (e.g.,
PMFS~\cite{PMFS}) manages the NVMM space and allows users to create and manage
files that reside in the NVMM.

\DAChell{} is implemented as an user space library, \libd{}, that transparently interposes on file access system calls.
\DAChell{} exploits DAX \texttt{mmap()} to make common-case access to file data
fast.  When an application opens a file, \DAChell{} intercedes, and
\texttt{mmap()}s the file's contents into the application's address space.
This provides direct access to the file's data and transfers the permission and
extent information from the file system's data structures into the
application's page table, leveraging the processor's fast memory
protection hardware to locate the file's data and protect its contents.

When the application calls \texttt{read()} and \texttt{write()}, \DAChell{}
translates them into calls to \texttt{memcpy()} to transfer data between
application buffers and the memory mapped file.  Other operations such as
\texttt{lseek()} simply update the state (e.g., the current file pointer) that
\DAChell{} maintains about each open file.  Consequently, common-case accesses
require no calls to the operating system, avoiding costly system calls and 
operating system and file system overheads.

\DAChell{} cannot eliminate all interactions with the operating system.  In
particular, any operations that modify file system meta data still must enter
the OS.  This means that \DAChell{} helps performance less for appends than for
normal write operations, since it needs to make a system call to extend both
the file and the in-memory mapping.  
Also, the updates to the file's modification time do not occur as they
would with normal accesses.  However, many performance-intensive applications
already disable file modification and access time updates to improve
performance.

Below we discuss the POSIX compatibility and implementation details of
\DAChell{}.

\subsection{POSIX Emulation}

\DAChell{} mimics the behavior of the normal POSIX file
interface. It also
enforce access restrictions (e.g., disallowing \texttt{write()} calls if the
file opened read-only, even if the file's permissions allow modification).
Implementing this functionality requires \DAChell{} to duplicate much of the
information that the kernel would usually manage.  This includes file
descriptor permission information, the close-on-exec flag, file position
information, and file descriptor aliasing information. For file descriptors
that point to files that don't reside in NVMM or that represent
network sockets and other resource, \DAChell{} passes requests to the default
POSIX functions that perform normal system calls.

\DAChell{} avoids the need to modify application source code or recompile by
using LD\_PRELOAD.  This allows \DAChell{} transparently link into the
applications and interpose on calls to libc.

\subsection{Implementation}

After the POSIX emulation determines which data an read or write targets, the
\DAChell{} translates the file operation into an operation on the memory mapped
contents of the file.

For each file being accessed via \DAChell{}, \DAChell{} divides the file into
2~MB \emph{chunks}.  A B-tree store information about the file's
\texttt{mmap()ed} chunks.  The key is the file offset, and the value is the
mapped address and length.  For each access, \DAChell{} queries the tree for
the mapping information.  If it finds the data, it performs a \texttt{memcpy()}
to transfer data between application buffers and the memory mapped NVMM pages.

If the file offset is not present in the B-tree, \DAChell{} uses
\texttt{mmap()} to map the data into the application's addresss space. Then,
it updates the B-tree and performs the \texttt{memcpy}.

\DAChell{} \texttt{mmap()s} 2~MB at a time, because \texttt{mmap()} is expensive
and when the file size changes, it may be necessary to remap the entire file. 
Mapping two megabyte chunks amortizes the cost of mapping and reduces the need
for remapping.  It also limits the number of memory-mapped regions the kernel
must maintain. Each \texttt{mmap()} request allocates a
virtual memory area (vma) in application's address space, and Linux kernel
limits each process to have up to 65,536 vmas. If the requests are small and
frequent, \texttt{mmap()}ing each of them will exhaust this number very quickly.
For small requests, the 2~MB \texttt{mmap()} also works as a prefetching
mechanism, reduce the overhead for small sequential requests.

\DAChell{} issues \texttt{mmap()} with the MAP\_POPULATE flag set to force the kernel to
populate the application's page table immediately.  Without the flag, the first
access to each page causes a page fault, degrading performance.

\DAChell{} relies on the DAX-style behavior of \texttt{mmap()} that maps NVMM
directly into the application's address space.  DAX-like functionality has
existed in some file systems for many years.  It used to be called
\emph{eXecute In Place}, or XIP, and mostly found use in embedded systems that
needed to execute
code directly from NOR flash devices.  DAX capabilities are already present in
PMFS~\cite{PMFS} and efforts are underway to add it to other file sysetms~\cite{ext4dax}.

With micro- and macro-benchmark, We found that \DAChell{} improves 4~kB access
latency by up to 19\% relative to file access through a PMFS, and improves
Berkeley-DB performance by 6\% over PMFS.
